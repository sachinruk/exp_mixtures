\documentclass{article}
\usepackage[]{algorithm2e}
\usepackage{algpseudocode}
\usepackage{amsmath}
\usepackage{amssymb}
\usepackage{cleveref}
\usepackage{mathtools}
\usepackage{graphicx}
\usepackage{caption}
\usepackage{subcaption}
\usepackage{bm}
%\usepackage{cases}

\newcommand{\cN}{{\mathcal{N}}}
\newcommand{\cAL}{{\mathcal{AL}}}
\newcommand{\cH}{{\mathcal{H}}}
\newcommand{\cGa}{{\text{Ga}}}
\newcommand{\cBe}{\text{Be}}

\newcommand{\cR}{{\mathcal{R}^\mathcal{D}}}
\newcommand{\cf}{{\mathbf{f}}}
\newcommand{\ce}{{\mathbf{e}}}
\newcommand{\cfs}{\mathbf{f_\star}}
\newcommand{\cx}{\mathbf{x}}
\newcommand{\cX}{\mathbf{X}}
\newcommand{\cxs}{\mathbf{x}_\star}
\newcommand{\ct}{{\mathbf{t}}}
\newcommand{\ceps}{{\mathbf{\epsilon}}}
\newcommand{\cI}{\mathbf{I}}
\newcommand{\cA}{\mathbf{A}}
%\newcommand{\cx}{{\mathbf{x}}}
\newcommand{\cB}{\mathbf{B}}
\newcommand{\cD}{\mathbf{D}}
\newcommand{\czero}{{\mathbf{0}}}
\newcommand{\cK}{{\mathbf{K}}}
\newcommand{\cSigma}{{\mathbf{\Sigma}}}
\newcommand{\cem}{{\mathbf{\m}}}
\newcommand{\cmu}{{\bm{\mu}}}
\newcommand{\ckappa}{{\bm{\kappa}}}
\newcommand{\ctheta}{\bm{\theta}}
\newcommand{\cexp}{{\text{exp}}}
\newcommand{\csI}{\sigma^2\cI}
\newcommand{\cts}{\cf}
\newcommand{\cData}{\mathcal{D}}
\newcommand{\cPhi}{\bm{\Phi}}
\newcommand{\Pzun}{p(z)}
\newcommand{\cpt}{\int_0^{\infty}\Pzun p(\cy|z)\,dz}
\newcommand{\cptb}{\int_0^{\infty}\Pzun~\cN(\cy| \czero,z\cK_{..})\,dz}
\newcommand{\cBess}{\mathcal{K}}
\newcommand{\cBessI}{\mathcal{I}}
\newcommand{\cint}{{\int_0^\infty}}
\newcommand{\cfKinvf}{\cf^T\cK^{-1}\cf}
\newcommand{\cGIG}{\text{GIG}}
\newcommand{\cpi}{\bm{\pi}}

\newcommand{\GIG}{{\mathbf{GIG}}}
\newcommand{\cEz}{\left\langle\frac{1}{z}\right\rangle_{q(z)}}
\newcommand{\cEzz}{\left\langle\frac{1}{z}\right\rangle}
\newcommand{\cEfKinvf}{\left\langle\cfKinvf\right\rangle_{q(\cf)}}
\newcommand{\czKsI}{\frac{1}{\cEzz_{q(z)}}\cK+\csI}
%angles paranthesis
\newcommand{\clangle}{\left\langle}
\newcommand{\crangle}{\right\rangle}
\newcommand{\chalf}[1]{\frac{1}{2#1}}
\newcommand{\css}{\sigma^2}
\newcommand{\clrbracket}[1]{\left(#1\right)}
\newcommand{\clrangle}[1]{\clangle#1\crangle}

\newcommand{\ctKinvt}{\cy^T\cK_{..}^{-1}\cy}
\newcommand{\cp}{\tilde{p}}
\newcommand{\ca}{\tilde{\alpha}}
\newcommand{\cat}{\mathbf{a}}
\newcommand{\cbt}{\mathbf{b}}
\newcommand{\cb}{\tilde{\beta}}
\newcommand{\cabt}{\ca\cb}

\newcommand{\cU}{\mathbf{U}}
\newcommand{\czKcsI}{z \cK+\csI}
\newcommand{\cKx}{\cK_{\cx,\cx}}
\newcommand{\cKsx}{\cK_{\cx_\star,\cx}}
\newcommand{\cKss}{\cK_{\cx_\star,\cx_\star}}
\newcommand{\cv}{\mathbf{v}}
\newcommand{\cz}{\mathbf{z}}

\newcommand{\cS}{\mathbf{S}}
\newcommand{\cw}{\mathbf{w}}
\newcommand{\cy}{\mathbf{y}}
\newcommand{\calpha}{\bm{\alpha}}
\newcommand{\clambda}{\bm{\lambda}}
\newcommand{\cgamma}{\bm{\gamma}}
\newcommand{\ceta}{\bm{\eta}}
\newcommand{\cbeta}{\bm{\beta}}

\newcommand{\caexp}{\alpha(1-\alpha)}
\newcommand{\csum}{\sum_{i=1}^{N}}
\newcommand{\csumt}{\sum_{t=1}^{T}}
\newcommand{\csumn}{\sum_{n=1}^{N}}

\newcommand{\vmf}{\text{vMF}}
\newcommand{\ga}{\text{Ga}}

\newcommand{\cfig}[3]{
\begin{figure}[h]
\centering
\includegraphics[]{#1}
\caption{#2}
\label{fig:{#3}}
\end{figure}
}

%For Dirichlet process
\newcommand{\cV}{\bigcup_{i=1}^{\infty}v_i}

%Didis commands
\newcommand{\ie}{{\sl i.e.}}
\newcommand{\eg}{{\sl e.g.}}
\newcommand{\etc}{{\sl etc.}}
\newcommand{\etal}{{\sl et al.}}

\newcommand{\cVprod}{\bigcup_{i=1}^{\infty}V_i}
\newcommand{\cVV}{\mathbf{V}}

\newcommand{\norm}[1]{\left\lVert#1\right\rVert}

\DeclareMathOperator*{\argmin}{\arg\!\min}

\newcommand{\cone}{\mathbf{1}}
\newcommand{\diag}{\text{diag}}
\newcommand{\Tr}{\text{Tr}}

\title{Dirichlet Process}
\date{}
\begin{document}
\maketitle

\section{MCMC method}
Model:
\begin{align}
\lambda_i\propto\frac{1}{\lambda_i}\qquad \lambda_i\in[a,b]\\
p(z_n=k|z_{/n},\alpha)=\begin{cases}
\frac{n_k}{N-1+\alpha} & \text{if }k\le K\\
\frac{\alpha}{N-1+\alpha} & \text{if }k=K+1\\
\end{cases}\\
p(x_n|\mathbf{\lambda},z_n)=\lambda_{z_n}\exp(-\lambda_{z_n}x_n)
\end{align}
where $K$ is the number of clusters which is currently occupied.

The posteriors on the given model are as follows. For $z_n$:
\begin{align}
\nonumber & \tilde{p}(z_n=k|z_{/n},x_n,\mathbf{\lambda})\propto 
\begin{cases}
\lambda_{k}\exp(-\lambda_{k}x_n) n_k\quad & \text{if }k\le K\\
\lambda_{k}\exp(-\lambda_{k}x_n) \alpha\quad & \text{if }k= K+1
%\exp{x} & \text{if } x \geq 0 \\
%   1       & \text{if } x < 0
\end{cases}\\
 \therefore & p(z_n=k|z_{/n},x_n,\mathbf{\lambda})=\frac{\tilde{p}(z_n=k|z_{/n},x_n,\mathbf{\lambda})}{\sum_{k=1}^{K+1}\tilde{p}(z_n=k|z_{/n},x_n,\mathbf{\lambda})}
\end{align}


For $\lambda_t$:
\begin{align}
\nonumber & p(\lambda_t|\cx,\cz)=\prod_{n=1}^{N}\clrbracket{\lambda_t\exp(-\lambda_t x_n)}^{1(z_n=t)}\frac{1}{\lambda_t}\\
& p(\lambda_t|\cx,\cz)=\lambda^{n_t-1}\exp\clrbracket{-\lambda_t\sum_{n\in t} x_n}\qquad \lambda_t\in [a,b]
\end{align}

\begin{algorithm}
%    \SetAlgoLined
    \KwData{$\alpha$,$\cy$}
    \KwResult{$\cz Chain,\clambda$.}
    $\lambda_t\sim Ga(1,1) \qquad t\in{1,\cdots,T}$\\
    $z_n\sim Cat([1,\cdots,1]/T) \qquad n\in{1,\cdots,N}$\\
    \For{i=1:iterations}{
    \For{n=1:N}{
    $z_n\sim p(z_n|z_{/n},x_n,\mathbf{\lambda})$\\
    }
    $\mathbf{n}=sum(\cz)$\\
    $\lambda_t\sim p(\lambda_t|\cx,\cz)$
    }
\end{algorithm}
where $T$ is a large value such that it is much larger than the possible number of clusters.

\section{Variational Bayes Method}
Model:
\begin{align}
\alpha \sim & \, Ga(\delta,\delta)
\label{eq:alpha}\\
v_i|\alpha \sim & \, Be(1,\alpha)\quad i\in\{1,...,T\}
\label{eq:v}\\
\lambda_i\sim & \, Ga(\delta,\delta)\quad i\in\{1,...,T\}\\
%\kappa_i|a,b\sim & \, Ga(\delta,\delta) \quad i\in\{1,...,\infty\}\\
\mathbf{z}_n|\mathbf{v}\sim & \, Cat(\cpi(\mathbf{v})) \quad n\in\{1,...,N\} \\
\mathbf{x}_n|\mathbf{z}_n\sim & \, exp(\lambda_{z_n}) \quad n\in\{1,...,N\}
\label{eq:gen}
\end{align}
where $\delta$ is taken to be a small value such that it portrays uninformative priors, $Cat$ is a categorical distribution with parameters $\cpi$. The elements $\cpi_i$ are a function of $\mathbf{v}$ such that 
\begin{align}
\cpi_i=v_i\prod_{j=1}^{i-1}(1-v_j)
\end{align}
This is re-parametrised in equation \ref{eq:z}.  In this case we consider the truncated DP where the number of clusters is limited by $T$, which is a large number. This implies that $v_{T+1}=1$.

Some handy reformulations:
\begin{align}
p(z_n=t|\mathbf{v})=& \prod_{i=1}^{T}(1-v_i)^{1((z_n=t)>i)}v_i^{1((z_n=t)=i)}\label{eq:z}\\
p(x_n|z_n=i)= &\prod_{n=1}^{N} \clrbracket{\lambda_i\exp(-\lambda_i x_n)}^{1(z_n=i)}
\end{align}

\textbf{For $\alpha$},
\begin{align}
\nonumber&\prod_{i=1}^{T}p(v_i|\alpha)p(\alpha)\\
\nonumber\log q(\alpha)\propto &  \sum_{i=1}^{T}\clrbracket{ \log\clrbracket{\frac{\Gamma(1+\alpha)}{\Gamma(1)\Gamma(\alpha)}}+\alpha\clrangle{\log (1-v_i)}}+(\delta-1)\log \alpha -(\delta)\alpha\\
\propto & (T+\delta-1)\log \alpha-\clrbracket{\delta-\sum_{i=1}^{T} \clrangle{\log (1-v_i)}} \alpha
\end{align}
Thus we have a gamma distribution on $\alpha$. The expected value $\clrangle{\alpha}=\frac{T+\delta}{\delta-\sum_{i=1}^{T} \clrangle{\log (1-v_i)}}$.

\textbf{For $v_i $}
\begin{align}
\nonumber& \prod_{n=1}^{N}p(z_n|\mathbf{v})p(v_i|\alpha)\\
\nonumber \log q(v_i)\propto & \csumn \clrangle{\log p(z_n|\mathbf{v})}+\clrangle{\log p(v_i|\alpha)}\\
\nonumber \propto&\sum_{n=1}^{N}\clrangle{1(z_n>i)}\log(1-v_i)+\clrangle{1(z_n=i)}\log(v_i)+(\clrangle{\alpha}-1)\log (1-v_i)\\
q(v_i) \propto & \,v_i^{\gamma_{1i}-1}(1-v_i)^{\gamma_{2i}-1}
\end{align}
where $\gamma_{1i}=\sum_{n=1}^{N} q(z_n=i)+1$ and $\gamma_{2i}=\sum_{n=1}^{N} q(z_n>i)+\clrangle{\alpha}$. To understand the $1(z_n>i)$ term, note that the term $(1-v_i)$ is only `activated' once the cluster assignment $z_n$ considered is for $i+1$ and above. Thus we have a beta distribution. The expectations $\clrangle{\log v_i}=\psi(\gamma_{1i})-\psi(\gamma_{1i}+\gamma_{2i})$ and $\clrangle{\log 1-v_i}=\psi(\gamma_{2i})-\psi(\gamma_{1i}+\gamma_{2i})$ are required.

\textbf{For $z_n$}
\begin{align}
\nonumber&\prod_{n=1}^{N} p(x_n|z_n)p(z_n|\cv)\\
\nonumber\log q(z_n)\propto & 1(z_n=t)\clrangle{\log p(x_n|\lambda_t)}+\sum_{i=1}^{T} 1((z_n=t)>i)\clrangle{\log(1-v_i)}+1(z_n=t)\clrangle{\log v_i}\\
\log q(z_n=t)\propto &\clrbracket{z_n=t}\clrbracket{\clrangle{\log \lambda_t}-\clrangle{\lambda_t}x_n+\clrangle{\log(v_t)}+\sum_{i=1}^{t-1}\clrangle{\log(1-v_i)}}
\end{align}
A multinomial is obtained for each observation $x_n$. Let $\pi_{n,t}\equiv q(z_n=t)$ and $\tilde{\pi_{n,t}}=\clrangle{\log \lambda_t}-\clrangle{\lambda_t}x_n+\clrangle{\log(v_t)}+\sum_{i=1}^{t-1}\clrangle{\log(1-v_i)}$. Thus,
\begin{align}
\pi_{n,t}=\frac{\tilde{\pi_{n,t}}}{\sum_{t=1}^{T}\tilde{\pi_{n,t}}}
\end{align}

\textbf{For $\lambda_t$}
\begin{align}
\nonumber & \prod_{n=1}^{N}p(x_n|z_n=t,\lambda_t) p(\lambda_t)\\
\nonumber & \log q(\lambda_t) \propto \sum_{n=1}^{N}q(z_n=t)\clrbracket{\log\lambda_t-\lambda_t x_n}+(\delta-1)\log\lambda_t-\delta\lambda_t\\
& \log q(\lambda_t) \propto \clrbracket{\csumn q(z_n=t) +\delta -1}\log\lambda_t-\clrbracket{\csumn q(z_n=t)x_n+\delta}\lambda_t
\end{align}
A gamma distribution is obtained where the expectations $\clrangle{\lambda_t}=\frac{\csumn q(z_n=t) +\delta}{\csumn q(z_n=t)x_n+\delta}$ and $\clrangle{\log\lambda_t}=\psi(\csumn q(z_n=t) +\delta)-\log(\csumn q(z_n=t)x_n+\delta)$.

The above variational distributions are iterated through till convergence of the lower bound:
\begin{align}
\nonumber lb=&\clrangle{\sum_{n=1}^{N}\csumt 1(z_n=t)\clrbracket{\log p(x_n|\lambda_t)-\log(v_t)+\sum_{i=1}^{t-1}\log(1-v_i)}}\\
\nonumber &+\csumt \clrangle{\log p(\lambda_t)+(\alpha-1)\log(1-v_t)}+(T+\delta-1)\clrangle{\log\alpha}-\delta\clrangle{\alpha}\\
\nonumber &-\clrangle{\csumt\clrbracket{\log q(\lambda_t)-\log q(v_t)}}-\clrangle{\csumn\csumt 1(z_n=t)\log q(z_n=t)}\\
&-\clrangle{\log q(\alpha)}
\end{align}

This algorithm is restarted with random parameter initialisations and the restart with the highest lower bound is chosen.

\clearpage
\section{Results}
The following synthetic experiments were performed:

\begin{algorithm}
%    \SetAlgoLined
    \KwData{$\alpha$,$N$}
    \KwResult{$\cy$.}
    $\theta_1\sim \cGa(1,0.1) \qquad i\in{1,\cdots,N}$\\
    $y_1\sim \cGa(1, \theta_1)$\\
    $\mathbf{n}=[1]$\\
    \For{$ i =2:iterations $}{
        $idx \sim Cat \clrbracket{\frac{1}{i-1+\alpha}[\mathbf{n}\quad \alpha]}$\\
        $y_i\sim \cGa(1, \theta_{idx})$\\
        $\mathbf{n}=\mathbf{n}+idx$
    }
\end{algorithm}

%\clearpage
Experiment 1: $\alpha=1$ and $N=100$. Results of segmentation is in Figure \ref{fig:N100}.
\begin{figure}[h]
\centering
\begin{subfigure}{0.4\textwidth}
\includegraphics[height=1.5in]{nmi100}
\caption{}
\end{subfigure}
\begin{subfigure}{0.4\textwidth}
\includegraphics[height=1.5in]{ari100}
\caption{}
\end{subfigure}
\caption{NMI and ARI metrics respectively. Red line is metric obtained by VB while blue trace plot is MCMC algorithm.}
\label{fig:N100}
\end{figure}
The number of clusters inferred by the MCMC DP algorithm is shown in Figure \ref{fig:z100}. The true number of clusters is 7.
\begin{figure}[h]
\centering
\begin{subfigure}{0.4\textwidth}
\includegraphics[height=1.5in]{z100}
\caption{}
\end{subfigure}
\begin{subfigure}{0.4\textwidth}
\includegraphics[height=1.5in]{z100_hist}
\caption{}
\end{subfigure}
\caption{}
\label{fig:z100}
\end{figure}

Experiment 2: $\alpha=1$ and $N=4000$. Results of segmentation is in Figure \ref{fig:N4000}. The true number of clusters is 8. The histogram of number of clusters inferred is shown in Figure \ref{fig:z4000}
\begin{figure}[h]
\centering
\begin{subfigure}{0.4\textwidth}
\includegraphics[height=1.5in]{nmi}
\caption{}
\end{subfigure}
\begin{subfigure}{0.4\textwidth}
\includegraphics[height=1.5in]{ari}
\caption{}
\end{subfigure}
\caption{NMI and ARI metrics respectively. Red line is metric obtained by VB while blue trace plot is MCMC algorithm.}
\label{fig:N4000}
\end{figure}
\begin{figure}[h]
\centering
\begin{subfigure}{0.4\textwidth}
\includegraphics[height=1.5in]{z4000}
\caption{}
\end{subfigure}
\begin{subfigure}{0.4\textwidth}
\includegraphics[height=1.5in]{z4000_hist}
\caption{}
\end{subfigure}
\caption{}
\label{fig:z4000}
\end{figure}

\clearpage
For the next set of experiments the data was generated as follows:
\begin{algorithm}
%    \SetAlgoLined
    \KwData{$\cpi$,$\clambda$, $N$}
    \KwResult{$\cy$}
%    $\theta_1\sim \cGa(1,0.1) \qquad i\in{1,\cdots,N}$\\
%    $y_1\sim \cGa(1, \theta_1)$\\
%    $\mathbf{n}=[1]$\\
    \For{$ i =1:N $}{
        $idx \sim Cat \clrbracket{\cpi}$\\
        $y_i\sim \cGa(1, \lambda_{idx})$\\
%        $\mathbf{n}=\mathbf{n}+idx$
    }
\end{algorithm}

Experiment 3: $\cpi=[0.4\quad 0.6]$, $\clambda=[2\quad 10]$, $N=50$.
\begin{figure}[h]
\centering
\begin{subfigure}{0.4\textwidth}
\includegraphics[height=1.5in]{nmi50_2}
\caption{}
\end{subfigure}
\begin{subfigure}{0.4\textwidth}
\includegraphics[height=1.5in]{ari50_2}
\caption{}
\end{subfigure}
\caption{NMI and ARI metrics respectively}
\label{fig:N50}
\end{figure}
\begin{figure}[h]
\centering
\begin{subfigure}{0.4\textwidth}
\includegraphics[height=1.5in]{z50_2}
\caption{}
\end{subfigure}
\begin{subfigure}{0.4\textwidth}
\includegraphics[height=1.5in]{z50_2_hist}
\caption{}
\end{subfigure}
\caption{}
\label{fig:z50}
\end{figure}

\clearpage
Experiment 3: $\cpi=[0.4\quad 0.6]$, $\clambda=[2\quad 10]$, $N=1000$.
\begin{figure}[h]
\centering
\begin{subfigure}{0.4\textwidth}
\includegraphics[height=1.5in]{nmi1000_2}
\caption{}
\end{subfigure}
\begin{subfigure}{0.4\textwidth}
\includegraphics[height=1.5in]{ari1000_2}
\caption{}
\end{subfigure}
\caption{NMI and ARI metrics respectively}
\label{fig:N1000}
\end{figure}
\begin{figure}[h]
\centering
\begin{subfigure}{0.4\textwidth}
\includegraphics[height=1.5in]{z1000_2}
\caption{}
\end{subfigure}
\begin{subfigure}{0.4\textwidth}
\includegraphics[height=1.5in]{z1000_2_hist}
\caption{}
\end{subfigure}
\caption{}
\label{fig:z1000}
\end{figure}

\end{document}